\documentclass{beamer}

\mode<presentation> {
	
	% The Beamer class comes with a number of default slide themes
	% which change the colors and layouts of slides. Below this is a list
	% of all the themes, uncomment each in turn to see what they look like.
	
	%\usetheme{default}
	%\usetheme{AnnArbor}
	%\usetheme{Antibes}
	%\usetheme{Bergen}
	%\usetheme{Berkeley}
	%\usetheme{Berlin}
	%%%\usetheme{Boadilla}
	%\usetheme{CambridgeUS}
	%\usetheme{Copenhagen}
	%%%\usetheme{Darmstadt}
	%%%\usetheme{Dresden}
	\usetheme{Frankfurt}
	%\usetheme{Goettingen}
	%\usetheme{Hannover}
	%\usetheme{Ilmenau}
	%\usetheme{JuanLesPins}
	%\usetheme{Luebeck}
	%\usetheme{Madrid}
	%\usetheme{Malmoe}
	%%%\usetheme{Marburg}
	%%%\usetheme{Montpellier}
	%\usetheme{PaloAlto}
	%\usetheme{Pittsburgh}
	%\usetheme{Rochester}
	%\usetheme{Singapore}
	%\usetheme{Szeged}
	%\usetheme{Warsaw}
	
	% As well as themes, the Beamer class has a number of color themes
	% for any slide theme. Uncomment each of these in turn to see how it
	% changes the colors of your current slide theme.
	
	%\usecolortheme{albatross}
	%%%\usecolortheme{beaver}
	%\usecolortheme{beetle}
	%\usecolortheme{crane}
	%\usecolortheme{dolphin}
	%\usecolortheme{dove}
	%\usecolortheme{fly}
	%\usecolortheme{lily}
	%\usecolortheme{orchid}
	%\usecolortheme{rose}
	%\usecolortheme{seagull}
	%\usecolortheme{seahorse}
	%\usecolortheme{whale}
	%\usecolortheme{wolverine}
	
	%\setbeamertemplate{footline} % To remove the footer line in all slides uncomment this line
	\setbeamertemplate{footline}[page number] % To replace the footer line in all slides with a simple slide count uncomment this line
	
	\setbeamertemplate{navigation symbols}{} % To remove the navigation symbols from the bottom of all slides uncomment this line
}


\usepackage[utf8]{inputenc}
\usepackage[ukrainian]{babel}

\usepackage{graphicx} % Allows including images
\usepackage{booktabs} % Allows the use of \toprule, \midrule and \bottomrule in tables

%----------------------------------------------------------------------------------------
%	TITLE PAGE
%----------------------------------------------------------------------------------------


\title[Short title]{Розв'язування задачі Діріхле-Неймана для рівняння Лапласа} % The short title appears at the bottom of every slide, the full title is only on the title page

\author{Бугрії Богдан, Середович Віктор} % Your name
\institute[UCLA] % Your institution as it will appear on the bottom of every slide, may be shorthand to save space
{
	Львівський національний університет імені Івана Франка \\
	Факультет прикладної математики та інформатики 
}
\date{\today} % Date, can be changed to a custom date

\begin{document}
	%------------------------------------------------
	
	\begin{frame}
		\titlepage
	\end{frame}
	
	%------------------------------------------------
	
	\begin{frame}
		\frametitle{Overview}
		\tableofcontents
	\end{frame}

	%------------------------------------------------

	\section{Постановка задачі} 
	
	\begin{frame}
		
	\end{frame}

	%------------------------------------------------

	%------------------------------------------------
	
	\section{Постановка задачі} 
	
	\begin{frame}
		
	\end{frame}
	
	%------------------------------------------------
	
	%------------------------------------------------
	
	\section{Постановка задачі} 
	
	\begin{frame}
		
	\end{frame}
	
	%------------------------------------------------
	
	%------------------------------------------------
	
	\section{Постановка задачі} 
	
	\begin{frame}
		
	\end{frame}
	
	%------------------------------------------------
	
	%------------------------------------------------
	
	\section{Постановка задачі} 
	
	\begin{frame}
		
	\end{frame}
	
	%------------------------------------------------

	
	%----------------------------------------------------------------------------------------
	%	Presentation examples
	%----------------------------------------------------------------------------------------
	
	\begin{frame}
		\frametitle{References}
		\footnotesize{
			\begin{thebibliography}{99} % Beamer does not support BibTeX so references must be inserted manually as below
				\bibitem[kress2012linear]{p1} Kress, R. (2012)
				\newblock Linear Integral Equations
				\newblock \emph{Applied Mathematical Sciences}
			\end{thebibliography}
		}
	\end{frame}
	
	%------------------------------------------------	
	
	\begin{frame}
		\frametitle{Blocks of Highlighted Text}
		\begin{block}{Block 1}
			some text some text some text some text some text some text some text some text some text some text some text some text some text some text some text some text 
		\end{block}
		
		\begin{block}{Block 2}
			some text some text some text some text some text some text some text some text some text some text some text some text some text some text some text some text 
		\end{block}
		
		\begin{block}{Block 3}
			some text some text some text some text some text some text some text some text some text some text some text some text some text some text some text some text 
		\end{block}
	\end{frame}
	
	%------------------------------------------------
	
	\begin{frame}
		\frametitle{Multiple Columns}
		\begin{columns}[c] % The "c" option specifies centered vertical alignment while the "t" option is used for top vertical alignment
			
			\column{.45\textwidth} % Left column and width
			\textbf{Heading}
			\begin{enumerate}
				\item Statement
				\item Explanationq
				\item Example
			\end{enumerate}
			
			\column{.5\textwidth} % Right column and width
			Lorem ipsum dolor sit amet, consectetur adipiscing elit. Integer lectus nisl, ultricies in feugiat rutrum, porttitor sit amet augue. Aliquam ut tortor mauris. Sed volutpat ante purus, quis accumsan dolor.
			
		\end{columns}
	\end{frame}
	
	%------------------------------------------------
	\section{Second Section}
	%------------------------------------------------
	
	\begin{frame}
		\frametitle{Table}
		\begin{table}
			\begin{tabular}{l l l}
				\toprule
				\textbf{Treatments} & \textbf{Response 1} & \textbf{Response 2}\\
				\midrule
				Treatment 1 & 0.0003262 & 0.562 \\
				Treatment 2 & 0.0015681 & 0.910 \\
				Treatment 3 & 0.0009271 & 0.296 \\
				\bottomrule
			\end{tabular}
			\caption{Table caption}
		\end{table}
	\end{frame}
	
	%------------------------------------------------
	
	\begin{frame}
		\frametitle{Theorem}
		\begin{theorem}[Mass--energy equivalence]
			$E = mc^2$
		\end{theorem}
	\end{frame}
	
	%------------------------------------------------
	
	\begin{frame}[fragile] % Need to use the fragile option when verbatim is used in the slide
		\frametitle{Verbatim}
		\begin{example}[Theorem Slide Code]
			\begin{verbatim}
				\begin{frame}
					\frametitle{Theorem}
					\begin{theorem}[Mass--energy equivalence]
						$E = mc^2$
					\end{theorem}
			\end{frame}\end{verbatim}
		\end{example}
	\end{frame}
	
	%----------------------------------------------------------------------------------------
	
\end{document} 
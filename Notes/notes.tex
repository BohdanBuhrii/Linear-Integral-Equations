%\documentclass[14pt,a4paper]{extreport}
\documentclass[12pt,a4paper]{article}
\usepackage[ukrainian]{babel}

\usepackage{amssymb}
\usepackage[active]{srcltx}
\usepackage[final]{pdfpages}

\usepackage[hidelinks]{hyperref}
%%%%%%%%%%%%%%%%%%%%%%%%%%%%%%%%%%%%%%%%%%%%%%%%%%%%%%%%%%%%%%%%%%
%\pagestyle{empty}                     %нумерацiя сторiнок i т.д.
\pagestyle{headings}                   %нумерацiя сторiнок вгорi зправа i т.д.
%\renewcommand{\baselinestretch}{1.5}   %мiжстрiчковий інтервал
%\parindent=7.5mm                      %абзацний відступ
 \righthyphenmin=2                     %перенос 2 останніх букв
 \pagenumbering{arabic}
 \tolerance=400
 \mathsurround=2pt
 \hfuzz=1.5pt
%%%%%%%%%%%%%%%%%%%%%%%%%%%%%%%%%%%%%%%%%%%%%%%%%%%%%%%%%%%%%%%%%%
 \hoffset=-0.5cm        %+2.5cm -- вiдступ вiд лiвого краю
 \voffset=-1.5cm        %+2.5cm -- вiдступ зверху
 \oddsidemargin=0.1cm   %ліве поле
 \topmargin=0.1cm       %верхнє поле
 \headheight=0.5cm      %висота верхнього колонтитулу
 \footskip=1cm          %висота нижнього колонтитулу
 \headsep=0.3cm         %відступ від колонт. до тексту
 \textwidth=17cm        %ширина сторінки
 \textheight=25.5cm     %висота сторінки
%%%%%%%%%%%%%%%%%%%%%%%%%%%%%%%%%%%%%%%%%%%%%%%%%%%%%%%%%%%%%%%%%%
 \newcounter{e}
 \setcounter{e}{0}
 \newcommand{\n}{\refstepcounter{e} (\arabic{e})}

% \setcounter{page}{1}
% \setcounter{section}{1}
%%%%%%%%%%%%%%%%%%%%%%%%%%%%%%%%%%%%%%%%%%%%%%%%%%%%%%%%%%%%%%%%%%
 \newcounter{stali}
 \setcounter{stali}{0}
 \newcommand{\s}{\refstepcounter{stali} \arabic{stali}}

 \newcommand{\st}{C_{\s}}
 \newcommand{\stl}[1]{C_{\s \label{#1}}}

 \newcommand{\cd}{{} $$ \vspace{-0.3cm} $$ {}}

%%%%%%%%%%%%%%%%%%%%%%%%%%%%%%%%%%%%%%%%%%%%%%%%%%%%%%%%%%%%%%%%%%

 \begin{document}
 \begin{LARGE}
 Нотатки
 \end{LARGE}
	
 \thispagestyle{empty}
 \section{Завдання}
 Знайти наближений розв'язок мішаної задачі Діріхле-Неймана для рівняння Лапласа у двозв'язній області методом ІР у випадку гладких границь, заданих параметрично. Використати потенціал простого шару. Чисельне розв'язування ІР здійснити методом колокації з використанням кусково-лінійних базисних функцій.
 
 \section{Рівняння Лапласа}
 $$
 \Delta u = 0
 $$
 
 \section{Область}
 \textbf{Гарасим:} Область краще малювати у вигляді боба, квасолі (не круглу, а з увігностями по боках)
 
 \section{Коректність}
 \textbf{З лекцій:} Дослідження коректності крайових задач здійснюється через дослідження відповідних їм ГІР другого роду. При цьому використовується доведена єдиність розв'язків крайових задач і теорії розв'язності інтегральних рівнянь (Рісса чи Фредгольма).



 Стійкість показувати на основі чисельних експериментів.


 \end{document}
